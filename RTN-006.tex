\documentclass[DM,authoryear,toc]{lsstdoc}

% lsstdoc documentation: https://lsst-texmf.lsst.io/lsstdoc.html

\input{meta}

% Package imports go here.

% Local commands go here.

%If you want glossaries
%\input{aglossary.tex}
%\makeglossaries

\title{Model for Community Science}

% Optional subtitle
% \setDocSubtitle{A subtitle}

\author{%
Melissa L. Graham, Christina L. Adair, Jim Annis, Jeff Carlin, Robert de~Peyster, Alex Drlica-Wagner, Gloria Fonseca Alvarez, Leanne P. Guy, Greg Madejski, and Andr\'es A. Plazas Malag\'on
}

\setDocRef{RTN-006}
\setDocUpstreamLocation{\url{https://github.com/rubin-observatory/rtn-006}}

\date{\vcsDate}

% Optional: name of the document's curator
% \setDocCurator{The Curator of this Document}

\setDocAbstract{%
A preliminary model for community science with the Rubin Observatory Legacy Survey of Space and Time (LSST).
}

% Change history defined here.
% Order: oldest first.
% Fields: VERSION, DATE, DESCRIPTION, OWNER NAME.
% See LPM-51 for version number policy.
\setDocChangeRecord{%
  \addtohist{1}{2021-01-31}{For milestone L3-CE-0020 "Develop a first model for community engagement for DP0.1".}{Melissa Graham}
  \addtohist{1.1}{2023-02-22}{Update name change to Community Science.}{Melissa Graham}
  \addtohist{1.2}{2023-10-31}{Rename User Acceptance Testing to Science Validation of the RSP.}{Andr\'es A. Plazas Malag\'on}
}

\begin{document}

% Create the title page.
\maketitle
% Frequently for a technote we do not want a title page  uncomment this to remove the title page and changelog.
% use \mkshorttitle to remove the extra pages

% ADD CONTENT HERE
% You can also use the \input command to include several content files.


\section{Introduction}\label{sec:intro}

The groundbreaking scientific potential of the Vera C. Rubin Observatory and the Legacy Survey of Space and Time (LSST), and the sheer volume and complexity of its data products, requires a proportionally innovative and progressive model for engaging and supporting the large and diverse community of scientists that it was designed to serve. 

This document was written by the Rubin Observatory pre-operations Community Science Team (CST; \S~\ref{sec:intro_org}) to provide an overview of the main strategies and activities that are in development to support scientists and students as they pursue their science goals with the LSST. 

Information about the Rubin Observatory's Education and Public Outreach (EPO) program, which seeks to engage the public in astronomical discovery and learning, can be found in PTSN-029. % \citeds{PTSN-029}.


\subsection{The Community Science Team (CST)}\label{sec:intro_org}

Within the organizational structure of Rubin Observatory Operations, the CST is part of the Rubin System Performance department (RPF).

In the context of Rubin Observatory, the term "system" refers to the complete end-to-end structure as a whole: the observatory site, telescope, and camera; the Legacy Survey of Space and Time (LSST); the Data Management System (DMS), including reduction, calibration, and validation of images, the resulting prompt and annual release data products, and the Rubin Science Platform (RSP); the documentation, infrastructure, and support for science; and the final scientific results. 

The System Performance department is responsible for ensuring that the LSST is proceeding with the efficiency and fidelity needed to achieve its 10-year science goals -- at minimum, those described in the Science Requirements Document \citedsp{lpm-17} -- and to also lead the field in emergent science.
The four teams within System Performance are Verification and Validation, Community Science, Survey Scheduling, and Systems Engineering.

The CST is part of the System Performance department because the scientific results produced by the community are a part of ``the system" and a key success metric of the Rubin Observatory.
Essentially, if the community is not producing scientific results, the system is not performing. 

As of Version 1 of this document, the CST has six members with a wide range of scientific expertise, practical knowledge of the Rubin Observatory data products and services, and experience in collaborative research and education.
The proposed staffing profile for the CST during Rubin Operations is summarized in \S~\ref{ssec:comp_staff}.


\subsection{Strategy}\label{ssec:intro_strat}

The Rubin Observatory and the LSST present a new regime not only in terms of its data volume and complexity, but also the size and variety of its science community.
It will be \textit{impossible} for Rubin staff to personally answer every question from the thousands of individuals working with Rubin data products and services.
Thus, the community science strategy focuses on making that kind of support \textit{unnecessary} by building and fostering a vibrant community, supported by infrastructure, that is able to help itself and to crowd-source solutions. 

The Rubin Observatory strategy for community science has three main components:
\begin{enumerate}
\item to facilitate access to and analysis of the data products via Rubin Observatory services,
\item to coordinate the expertise of the project staff and the science community, and
\item to empower a self-supporting and collaborative community.
\end{enumerate}

This model for community science has several guiding principles:
\begin{itemize}
\item access to documentation should be open and available to all communities
\item support for science should be scalable, sustainable, and equitable
\item interactions should prioritize inclusivity and respect at all times
\item entry points and materials for beginners should be clearly identified
\item changes to the system should be communicated clearly and promptly
\item scientists should be enabled to propose or contribute improvements to the Rubin system
\end{itemize}


\subsection{Support for Science vs. Technical Assistance}\label{ssec:intro_tech}

All of the components of this model for community science described in \S~\ref{sec:mod} are designed to support scientific endeavors, and not to provide technical support or service access issues.
For example, technical support issues pertaining to, e.g., account authentication (\textit{``Help, I forget my password"}) or access issues (\textit{``I'm getting a 504 Gateway Timeout when trying to log in to the RSP right now"}) would not be posted to \url{Community.lsst.org} (\S~\ref{ssec:mod_support}) -- a separate, RSP-related venue will be provided for such things.
Issues related to the scientific applications of the LSST data products, such as \textit{``I'm trying to insert simulated galaxies into images and their measured fluxes look wonky?"}, are the type of question that this model for community science is being designed to serve.

The CST recognizes that there might be grey areas here, and that people might forget about or not realize the different support resources for the different types of questions; either people will be politely redirected or the question can be passed between teams.


\subsection{Communities}\label{ssec:intro_comms}

In presenting a model for community science, it is essential to understand the various overlapping communities involved.

\begin{itemize}
\item \textbf{Rubin community} is a broad term that refers to anyone, anywhere, interacting with the Rubin Observatory data products and services, in any capacity. It is a union set of all of the following communities.
\item \textbf{Science community} refers specifically to the subset of the Rubin community doing scientific analyses with the LSST data products and services.
\item \textbf{Science Collaboration members} are science community members who have joined one of the LSST Science Collaborations.
\item \textbf{Rubin Observatory staff} are individuals who work for Rubin Observatory as part of the construction or operations teams. All staff are members of the Rubin community and are data rights holders. Any staff engaging in scientific analyses are part of the science community, and many staff are also Science Collaborations members.
\item \textbf{Data rights holders} are individuals with the right to access, analyze, and publish work based on the Rubin Observatory proprietary data products and services (such as the RSP \citedsp{lse-319}) as described in the Rubin Data Policy document, \citeds{rdo-013}. 
\end{itemize}

Note that the definitions of the Rubin and Science communities are not restricted to data rights holders. 
Although most individuals working in these communities will be data rights holders, the CST recognizes that individuals without data rights may do science with post-proprietary data products \citedsp{dmtn-144}.
The components of this model have been designed to accommodate this, such as the open access to the resources such as documentation or \url{Community.lsst.org} (\S~\ref{sec:mod}).

In the Data Policy \citedsp{rdo-013}, the terms "LSST user" and "user" specifically represent a data rights holder using the Rubin Observatory data products or services. 
However, in this document the term "user" does not necessarily mean a data rights holder, and the term "user" is used more generically as users of services.
For example, users of \url{Community.lsst.org} (does not require data rights), and users of the RSP (does require data rights).

In addition, the design of this model community science accommodates the fact that scientists are members of many other types of intersectional research communities such as career stage; institutional type; research methodologies (e.g., experimental, theoretical); messenger type or wavelength regime; location and language; scientific interests; and so on.


\subsubsection{LSST Science Collaborations}\label{sssec:intro_comms_scicoll}

The LSST Science Collaborations have a special place in this model for community science because they are already a community of very engaged scientists who actively welcome and recruit the participation of others.
The CST recognizes that the Science Collaborations will continue to be the first place many scientists turn to for support and engagement.
Thus, many aspects of the CST's model for community science are also designed to directly support the Science Collaborations.

Examples of the model's components that would directly support the Science Collaborations include: ensuring the documentation infrastructure can accept contributions from the science community (\S~\ref{ssec:mod_docs}); using a publicly-accessible online forum with groups and categories for the Science Collaborations (\S~\ref{ssec:mod_support}); attending Science Collaboration meetings (\S~\ref{ssec:mod_interface}); soliciting Science Collaboration expertise in the issue resolution process (\S~\ref{ssec:mod_coord}); seeking Science Collaboration representation on RSP user committees (\S~\ref{sssec:mod_coord_uc} and \ref{sssec:mod_coord_rac}); and facilitating the Science Collaboration Liaison Program (\S~\ref{sssec:mod_interface_SCliaison}).

By maintaining open lines of communication with the Science Collaborations, the CST will endeavor to ensure that our joint efforts in community science are constructive, mutually beneficial, and avoid redundancy.

The CST recommends joining one or more Science Collaborations as the best way to get involved in preparing for science with the LSST and to learn more about Rubin Observatory.
Readers who are unfamiliar with the LSST Science Collaborations can find a description at \url{https://www.lsst.org/scientists/science-collaborations}.



\section{Model Components}\label{sec:mod}

This section lists and describes the CST's planned, in-development, and in-progress activities for community science. 


\subsection{Documentation and Resource Materials}\label{ssec:mod_docs}

The CST will oversee and maintain the contents of documentation and resource (educational) materials that are primarily community-facing, such as:
\begin{itemize}
\item the Data Products Definitions Document \citedsp{lse-163}
\item user guides for the RSP aspects (Portal, Notebook, and API)
\item tutorials and demonstrations (e.g., Jupyter notebooks, database query examples)
\item content for scientists in the Rubin Observatory website
\item Rubin Tech Notes (RTN; such as this one)
\end{itemize}

The development of community-facing documentation will focus on identifying and serving the user's needs, and ensuring that written materials are accurate, complete, and appropriate for the audience for which it is intended.
Understanding what experienced and non-experienced users need from documentation will drive the depth of technical knowledge included in the resource materials and the type of documents created.
Soliciting feedback from users will help identify areas where resources should be created or updated.

In alignment with the CST's strategies (\S~\ref{ssec:intro_strat}), all documentation and materials will be openly available whenever possible (i.e., unless it contains proprietary data and should not be shared as specified by the Data Policy, \citeds{rdo-013}), and advanced-level materials should reference supporting materials to enable use by non-experts.

Resources and documentation which are not just community-facing but also used internally, such as source code documentation for the LSST Science Pipelines, would be created by, e.g., the Data Production team.
The CST will interface with the science community and with Rubin departments to provide feedback and help to ensure that such documentation is complete and kept up-to-date.

In keeping with the guiding principle (\S~\ref{ssec:intro_strat}) of enabling users to contribute, the CST will develop infrastructure to provide ways for user-generated documentation and resource materials to be shared.


\subsection{Scalable and Sustainable Support Infrastructure}\label{ssec:mod_support}

During Rubin Observatory operations, the thousands of scientists using the data products and services will have a wide variety of experience, expertise, and science goals.
It will be impossible for Rubin Observatory staff to answer all individual questions -- more specifically, a Help Desk type of interface for scientific and technical support would be inappropriate and inadequate as it could become a bottleneck.

Instead, the CST is building an online community to provide scientists with the ability to crowd-source solutions from a deep reservoir of collected expertise, and the CST will continue to nurture this community throughout operations.

The open-source Discourse community management platform, used by the construction-era Data Management team to build the Rubin Observatory Community Forum (\url{Community.lsst.org}), will continue to be used to build the Rubin Operations online community.
Technical maintenance of the forum will continue to be provided by the Data Production team in the operations-era. 

The Discourse platform has several advantages, as originally outlined in SQR-011: %\citeds{SQR-001}:
\begin{itemize}
\item native to the web (linkable; searchable; markdown format)
\item prioritizes forum user experience (threading; liking; notifications)
\item open platform (publicly viewable, accounts for all)
\item hierarchical categorization (easy to navigate when browsing)
\item support for Q\&A, and marking replies as 'solutions'
\item enables synchronous and asynchronous conversations
\end{itemize}

Use of the Community Forum for providing support for science to the Rubin community is described in more detail in \citeds{dmtn-155}.

% Other communication tools, such as Slack, will continue to be used by the CST for \textit{real-time} discussions with Rubin staff; for participation of Rubin community members in virtual or face-to-face meetings and workshops such as the Project and Community Workshop; and for communicating with the Science Collaborations (if that remains one of their preferred methods for communication).
% However, Slack will not be used as a venue for Q\&A or support for science, as it is neither scalable nor sustainable in the long-term.

The CST recognizes that modes for private Q\&A and issue reporting regarding sensitive scientific topics or proprietary data will also need to be supported.
The CST is currently looking into how to offer confidential support via the Community Forum.
As a reminder, a different channel will be provided for technical assistance, such as questions about RSP account services and authentication (\S~\ref{ssec:intro_tech}).

The CST recognizes that some of the Discourse functionality (such as account trust levels, badges, and tags) are not yet well understood by forum participants, and that some reorganization of the categories could improve the forum user experience.
The CST is currently reviewing existing functionality and planning upgrades to this service (and documenting this work with topics in the Meta category).
The CST will continue to evolve the forum in response to activity trends and feedback during the pre-operations data previews, so users should expect changes to the organizational structure and design. 

\subsubsection{Use of the Forum for scientific discussion}\label{sssec:forum_discussion}

Two high-level categories are available for scientific discussion: the Science and the Science Collaborations categories and their sub-categories. 
Categories for scientific discussions are not monitored by Rubin staff, and do not have the ``marked solution'' functionality activated.
Forum users may propose new sub-categories by making a new topic in the Meta category to argue why the sub-category is needed.
Forum administrators (Rubin staff) review and discuss the proposals for new categories and only admins can create new categories.

{\bf Science Collaborations}:
Advice for Science Collaboration chairs on best practices for the use of Forum categories and groups associated with their Science Collaboration is provided in the Meta topic on ``How to use the Forum as Science Collaboration chair.'' 

{\bf Time-domain research announcements}:
Forum users may propose to use the Discourse API in a sub-category to auto-generate new topics for time-domain phenomena.
This enables the Forum to be used for public, worldwide discussions about time sensitive follow-up. 
New APIs are first implemented on a year-long trial basis to confirm that the number of auto-generated new topics is on order a few per day or less (any more is too many for human interaction, which means the Forum is the wrong tool).
Such APIs must be muted from the Forum landing page feed, and it is recommended to restrict access to a group (i.e., users must opt-in to seeing the announcements), but the group itself can be openly joinable.

The primary use of the Forum is Rubin user support (helpdesk functionality), and the secondary use is Rubin news communications.
The Forum admins (Rubin staff) retain the right to restrict or remove any other functionality, e.g., if it interferes with users' ability obtain support or news.


\subsubsection{Use of Generative Artificial Intelligence (AI)}\label{sssec:use_ai}

The CST is exploring the potential of generative artificial intelligence (GAI) to enhance our support infrastructure.
Current work includes the development of a ``RubinGPT" or similar AI chatbot, leveraging advancements in natural language processing and machine learning for intuitive user support.
The CST is also considering integrating AI bots with our existing Discourse platform to improve user experience and allow our support staff to focus on more complex inquiries.
Considerations also include the assessments of scalability and sustainability of costs, as well as concerns related to accuracy, reproducibility, and the energy footprint of making inferences using GAI. 


\subsection{Live Interactions}\label{ssec:mod_interact}

Interacting live, virtually or face-to-face, with scientists using the Rubin Observatory data products and services will be an important component of the model for community science.

Although it is not always possible to reach a large audience with live interactions, targeted workshops to train scientists (including students) working with LSST data could be used to seed participation within the communities and groups mentioned in \S~\ref{ssec:intro_comms}.

This approach of seeding engagement with a limited number of participants in an interactive event is being taken with two initiatives occurring during pre-operations, at or before the time of Version 1 of this document: the Stack Club Course\footnote{For more information about the Stack Club see \url{https://github.com/LSSTScienceCollaborations/StackClub} and about its Course see \url{https://github.com/LSSTScienceCollaborations/StackClubCourse}.} and community participation in Data Preview 0\footnote{For more information about Data Preview 0 see \url{ls.st/clo4618}.}.
In both cases, a diversity-based participant selection process was implemented, and the CST intends to further refine such processes for future interactive events. 

The following list contains some examples of how the CST might interact with the science community.
\begin{itemize}
\item run topical workshops, sprints, hackathons, etc. in support of emergent science needs
\item attend conferences and meetings to promote the use of Rubin Observatory data, to interact with scientists using Rubin data, and to stay up-to-date with scientific progress
\item assist with running the annual Project and Community Workshop
\item conduct open virtual drop-in sessions to distribute information and/or host Q\&A 
\item chair the User Committee meetings (e.g., \S~\ref{sssec:mod_coord_uc})
\item attend Science Collaboration meetings as appropriate (e.g., \S~\ref{sssec:mod_interface_SCliaison})
\end{itemize}

For example, at the time of Version 1 of this document, the CST had helped to run the virtual Project and Community Workshop in August 2020\footnote{\url{https://project.lsst.org/meetings/rubin2020/}} (e.g., participating in the Science Organizing Committee); conducted a series of open virtual drop-in sessions about community engagement\footnote{\url{https://ls.st/clo4527}} as part of the preparation of this document; and developed materials\footnote{\url{https://www.lsst.org/aas237/operations_updates}} for and helped to staff the Rubin Observatory booth at the winter meeting of the American Astronomical Society in January 2021. 

Future topical workshops to address emergent science needs in the community might be similar to past workshops run successfully by the Data Management team about community alert brokers and image processing algorithms. 
For example, in 2023, the CST held a week-long virtual summer school with a focus on the DP0.2 dataset. 
Future virtual drop-in sessions will support grant proposal preparation. 

The CST anticipates that its members could become highly visible as ``Rubin insiders" who receive invitations to visit or to speak at institutions as a Rubin Observatory or CST representative, and that it may be wealthier institutions that are able to fund these visits.
In the spirit of the strategies in \S~\ref{ssec:intro_strat}, the CST is developing policies to ensure that diversity, equity, and inclusion are maintained for the limited resources available for face-to-face interactions. 

Current policies focus on strategies to encourage the inclusion of small or underserved institutions and maintaining access to resources by publicly sharing recordings of live interactions. 
For example, the CST has introduced procedures to request custom seminars. In addition to prioritizing invitations from small or underserved institutions, requesters are asked to either facilitate the participation of scientists and students from these institutions, such as funding their access to the talk or meeting, or to facilitate the recording and sharing of the talk or meeting by the CST. 
To encourage collaboration, particularly between different institution types, the CST is also developing strategies to identify suitable speakers among Rubin staff and the LSST Science Collaborations.

\subsection{Interfaces Within the Rubin Community}\label{ssec:mod_interface}

The CST will interface with many different groups within Rubin Observatory. 
This is a very brief overview describing some of these internal interfaces.
\begin{itemize}
\item Rubin System Performance (RPF): System Engineering (SE), Survey Scheduling (SS), and Verification and Validation (VV) -- regular RPF leadership team meetings, collaborative work organized and assigned in Jira\footnote{Jira is a software package developed by Atlassian, used internally by Rubin Observatory for issue tracking and work management.}.
\item Rubin Observatory Operations (ROO), Rubin Data Production (RDP) -- topical communications to solve problems; collaborative work organized and assigned in Jira.
\item Rubin Education and Public Outreach (REP): regular collaboration, especially regarding citizen science projects.
\item RSP developers: deliver feedback from Users Committee and User Acceptance Testing to guide development; help to communicate changes and updates to the science community.
\item Communications team: regular discussions to generate and maintain website contents, advertise initiatives and circulate announcements, co-organize the annual Rubin Project and Community Workshops, and coordinate presence at national conferences.
\item Director's Office -- keep apprised of progress, seek advisement as needed, incorporate feedback.
\item Science Advisory Committee (SAC) -- attend meetings when possible, keep apprised of progress and seek advisement as needed, incorporate suggestions into CST work.
\end{itemize}

The CST will also interface with groups that extend beyond, or are independent of, Rubin Observatory.
As with the above, the following list is meant to be demonstrative, not exhaustive.
\begin{itemize}
\item LSST Science Collaborations -- regular communications with the coordinator and chairs to keep them apprised of CST initiatives; keep apprised of open issues in the resolution process and opportunities to provide input; work together on common goals and joint initiatives (e.g., Data Previews, Research Inclusion); facilitate the liaison program (\S~\ref{sssec:mod_interface_SCliaison}).
\item In-Kind Contributors: assist with the ingestion of deliverables into the Rubin system.
\item NOIRLab: regular contact with the Communications team (e.g., for website contents), the DataLab team (e.g., for shared resource materials or initiatives like the Data Previews), and the Community Science Data Center (CSDC) to share resources for common goals, especially related to Research Inclusion initiatives.
\item LSSTC: maintain communications (e.g., present at board meetings when requested); collaborate on the shared goal of enhancing scientists' ability to secure funding.
\end{itemize}


\subsubsection{Science Collaboration Liaison Program}\label{sssec:mod_interface_SCliaison}

The CST will facilitate Rubin Observatory staff -- likely including several or all CST members -- to serve as ``liaisons" to each of the LSST Science Collaborations during Operations. 
During construction, this program is being administered by the Data Management System Science Team (DM-SST).

The CST will work with the LSST Science Collaborations to define the liaison's role, responsibilities, and term (examples below), and to set up a process to match Rubin staff with appropriate LSST Science Collaborations.
The CST will work with the managers in the Rubin departments to ensure that the work done by liaisons is tracked and accounted for in their hours (or Jira story points). 

Preliminarily, a Science Collaboration Liaison might have the following qualities:
\begin{itemize}
\item has scientific expertise that matches the Science Collaboration
\item has detailed knowledge about the Rubin data products, survey strategy, etc. 
\item is able to provide high-level scientific guidance 
\end{itemize}

Preliminarily, a draft set of responsibilities for liaisons might be:
\begin{itemize}
\item engage in regular interactions with the Collaboration
\item attend relevant meetings when possible (e.g., biweekly or monthly all-hands)
\item respond to mentions in the Collaboration's forum category or Slack channel
\item provide timely responses to questions from the Collaboration's chair(s)
\item accommodate requests from chair(s) for meeting attendance when possible
\item attend special topical sessions or Collaboration workshops, when possible
\item review specific wording (e.g., sections, paragraphs) of Collaboration publications or proposals that directly reference Rubin Observatory (e.g., the data rights policy, naming guidelines, operations timelines, data products definitions) to ensure it is correct (when possible and as necessary)
\item arrange for alternate Rubin staff to assist with above requests as needed
\item assist the CST with scientific investigations associated with the Collaboration, such as identifying Collaboration members with expertise to review science impacts of internally proposed changes to the System
\item alert the CST to science issues that may arise, and advise on issue priority, urgency, and potential mitigations or resolutions
\item ensure their work is tracked in Jira tickets and that their managers are aware of this role
\end{itemize}



\subsection{Coordinating Expertise}\label{ssec:mod_coord}

Coordinating expertise within the Rubin community, through the interfaces listed in \S~\ref{ssec:mod_interface}, will be an essential component of the model for community science.

The CST envisions coordinating expertise on both an \textit{emergent, as-needed} basis to meet problems or opportunities as they arise, and on a \textit{consistent, periodic} basis to prevent problems or identify opportunities well in advance.

For example, the CST will assist with the resolution of emergent issues and problems (\S~\ref{sssec:mod_coord_res} and help with opportunities to federate user-generated data products or algorithms into the Rubin system (\S~\ref{sssec:mod_coord_ug}).
The CST will also facilitate the Users Committee (\S~\ref{sssec:mod_coord_uc}) and the Resource Allocation Committee (\S~\ref{sssec:mod_coord_rac}), which would meet on a periodic basis.


\subsubsection{Issue Resolution}\label{sssec:mod_coord_res}

It is a certainty that questions about and problems with the Rubin Observatory data products and services will arise for which an answer or solution does not already exist or is not easy to derive. 
In these cases, the CST would assist with the issue-resolution process by coordinating expertise within the Rubin community and shepherding the issue towards a conclusion.
The CST would also take responsibility for defining the process to report issues, and ensuring that it is equally accessible to all members of the science community, and that the prioritization and resolution process is as transparent as possible.

However, it should be noted that the CST would not \textit{have to} be involved in all issue resolution processes; that situation could impose an unnecessary bottleneck.
As a specific example, a Science Collaboration liaison (\S~\ref{sssec:mod_interface_SCliaison}) who works in the Rubin Data Production department could identify an issue, discuss with their manager, instantiate a Jira ticket, and do work to resolve the issue.
If resolving the issue does not require community input, and if the resolution does not create a major change to a data product or service that would need to be widely announced, then there might be no need to involve the CST in the process.
In these cases, the CST encourages all parties to use of the Community forum to discuss and report resolved issues (\S~\ref{ssec:mod_support}).

The CST is assembling use-cases that envision \textit{specific} types of issues and their resolution process in \citeds{rtn-002}, to be ingested in to a model-based systems engineering framework in order to more thoroughly identify needed components and interfaces of this model.
As examples, \citeds{rtn-002} describes the workflow for issues in which a scientist identifies a new parameter for which detection efficiencies should be calculated in the LSST science pipelines, in which a camera fault requires modification of the alert production pipeline and rapid communication with scientists, and in which an RSP user encounters difficulties with large queries. 

The following list is an example workflow for a generic issue resolution process that involves the CST.
\begin{enumerate}
\item A scientific risk, problem, or opportunity is identified by, or brought to the attention of, the CST. Its origin could be from the science community or from Rubin staff.
\item The CST creates an internal Jira Issue Ticket assigned to the CST member with relevant expertise and adds relevant watchers (e.g., from Data Production). A priority and a deadline are assigned. 
\item The CST coordinates expertise from across the Rubin community, gathering information relevant to the issue, facilitating communications, and guiding the response.
\item Work proceeds to resolve the issue with participation from across the Rubin community, and intermediate communications and reprioritizations are facilitated by the CST.
\item After the issue is resolved, the CST closes the ticket and updates documentation as required. Major changes to data products or services are circulated to the science community.
\end{enumerate}


\subsubsection{Federating User-Generated Data Products or Algorithms}\label{sssec:mod_coord_ug}

During Rubin Observatory operations, the combined expertise of the science community will result in user-generated data products (e.g., value-added catalogs) or software (e.g., algorithms) that should be federated with or implemented into the Rubin system.

The CST will work with the Rubin Data Production department to develop policies and methods to identify such opportunities and to provide support during the federation process.

One example of this kind of process to enable the science community to contribute expertise and improve the Rubin system is the joint community-project roadmap for photometric redshift estimates for the data release object catalog, detailed in \citedsp{dmtn-049}.


\subsubsection{Users Committee}\label{sssec:mod_coord_uc}

The Users Committee -- a sub-committee of the Science Advisory Committee (SAC) -- will be charged with advising the Lead Community Scientist and the Rubin Observatory Director on the usability of LSST data products and services, especially the RSP, and advocate for science-driven improvements and changes to them.

As of Version 1 of this document, the charge for the Users Committee was being drafted by the SAC.
Membership appointments to the Users Committee would be made by the Rubin Observatory Director under advisement from the SAC.
Members will be data rights holders who use the RSP, and will represent the broad diversity of the global Rubin science community, including individuals from underrepresented groups in astronomy and underserved institutions.
The CST will set up the Users Committee's communications infrastructure and meetings, will ensure that their biannual reports are delivered to the Director, and will generate internal Jira tickets to, e.g., resolve issues or request new features, as needed (\S~\ref{sssec:mod_coord_res}).
The degree to which the Users Committee participates in Science Validation of the RSP (\S~\ref{sssec:mod_dev_sv}) remains to be determined.


\subsubsection{Resource Allocation Committee}\label{sssec:mod_coord_rac}

During Rubin Observatory operations, some individuals or groups of RSP users might require storage and/or computational processing resources in excess of the basic quota allocated to all accounts.
If so, the CST will set up and facilitate the Resource Allocation Committee (RAC).
The RAC would be analogous to a Time Allocation Committee (TAC) for a telescope, but in this case the limited resource would be compute cycles, disk space for storage, and potentially significantly large bulk data downloads.
The RAC would consider the scientific justification for increases to basic quota for RSP users and be advisory to the Operations Director.
Membership would include representatives from the science community, and Rubin Operations staff members would assist the RAC by assessing the technical feasibility of requests as appropriate.

It remains to be determined whether the RAC assesses urgent requests as they arise, i.e., on an emergent basis, and/or whether it considers requests on a periodic basis.



\subsection{Components In Development}\label{ssec:mod_dev}

As of Version 1 of this document, there are several other components of this model for community science which are still in very early development.


\subsubsection{Proactive Strategies for Diversity, Equity, and Inclusion}\label{sssec:mod_dev_proact}

The Rubin Observatory community science strategies prioritize diversity, equity, and inclusion (DEI) as guiding principles (\S~\ref{ssec:intro_strat}) for all of the model components in this section.

The CST is also developing more specific, proactive DEI strategies, many in collaboration with the other groups with which the CST interfaces (\S~\ref{ssec:mod_interface}), especially the LSST Science Collaborations, the Rubin Observatory Operations Executive team, NOIRLab, and the LSST Corporation.
Some of these strategies are being developed and tested in conjunction with Data Preview 0 \citedsp{rtn-004}.

\begin{itemize}
\item helping to formulate and implement Rubin-wide policies for research inclusion
\item helping to build research partnerships with scientists at small and underserved US institutions
\item prioritizing and facilitating participation from individuals who self-identify with underrepresented groups in astronomy
\item collecting and evaluating data about RSP users (and co-authors of Rubin-related journal articles) to understand which communities are accessing the LSST data products, and using this to inform further proactive DEI efforts
\end{itemize}

The CST acknowledges that DEI-focused strategies were also raised as issues that are very important to the science community during the CST's virtual open discussion series on community engagement in late 2020.
Future versions of this model for community science will feature more fully developed descriptions of these proactive, DEI-driven strategies.


\subsubsection{Science Validation of the RSP}\label{sssec:mod_dev_sv}

The CST will conduct Science Validation of the RSP and associated community-facing services at regular intervals.
Examples include prior to major updates to the RSP or before an annual Data Release.
The goal is to create, test, and execute a process to scientifically validate the RSP.
This process should be designed to answer questions about the RSP's functionality and usability.
Specifically, the validation process will aim to determine whether the RSP meets the scientific needs of its users, if the necessary tools exist, if users are able to perform scientific analyses in the RSP, and if the interface is inclusive of all users and potential users while being intuitive to use with minimal instruction.
An understanding the diversity of Rubin users and RSP use-cases is key to developing the process for science validation of the RSP.

The components and workflow for the RSP science validation process have yet to be defined, but might include:
\begin{itemize}
\item coordinating with the Users Committee (\S~\ref{sssec:mod_coord_uc}) to include RSP users from the community in RSP science validation campaigns,
\item contracting with prosessional user experience (UX) and user interface (UI) experts on accessibility,
\item reporting issues and coordinating with Rubin Observatory teams to resolve them, and
\item producing a science validation report to the Rubin Observatory product owners.
\end{itemize}


\subsubsection{Citizen Science Methodologies}\label{sssec:mod_dev_citizen}

Given the unique size and complexity of the LSST data set there will be certain scientific results that are only obtainable via citizen science methodologies.
The CST will help scientists identify and prepare LSST data and documentation for citizen science programs, and will coordinate with the EPO department and Zooniverse to help ensure that the data are installed and the outputs will meet the program's science goals.

The CST's activities related to citizen science are still in the early stages of development.






\section{Proposed Operations Work Packages}\label{sec:comp}

The following is a very brief summary of the proposed work packages and staffing profile for the CST, which have been generated in tandem with the development of Version 1 of this model for community science.


\subsection{Proposed Work Breakdown Structure}\label{ssec:comp_wbs}

As of Version 1 of this document, these are the five proposed components of the CST's work breakdown structure (WBS) packages.

\begin{enumerate}
\item \textbf{Coordinate Support for Science:} curate the community forum, guide scientists to help each other find solutions, identify emergent issues, and coordinate investigations to resolve them.
\item \textbf{Science Validation of the RSP:} Conduct RSP science validation campaigns in coordination with the Users Committee prior to major updates, report to the Rubin Observatory product owners and appropriate internal departments, and follow-up on any issues.
\item \textbf{Scientific Documentation:} oversee the contents and delivery of community-facing scientific documentation related to the LSST data products and software. 
\item \textbf{Interact with the Science Community:} promote the understanding and use of Rubin Observatory data products and software at meetings, engage in regular interactions with scientists in the community forum, and facilitate the Science Collaboration liaison program and the Users Committee.
\item \textbf{User-Generated Data Products:} review and guide requests to federate user-generated algorithms and data products, support scientists to generate data products for citizen science programs, and facilitate the Resource Allocation Committee.
\end{enumerate}

As of Version 1 of this document there is not a one-to-one match between the five work packages above and the model components described in the six subsections of \S~\ref{sec:mod}, however, all components are represented in the work packages (i.e., the CST is scoped for all the listed components).


\subsection{Proposed Staffing Profile}\label{ssec:comp_staff}

As of Version 1 of this document, the proposed CST staffing profile would ramp up to 10.5 full-time equivalent (FTE) positions by the start of Operations, and in 2021 the CST was staffed at 3.25 FTE. 
This proposed staffing profile in Operations would include the team lead (1 FTE), two documentation specialists (1 FTE), several community scientists covering the main LSST science pillars (7 FTE), and two citizen-science related positions (1.5 FTE). 
The source of these appointments will come from NOIRLab, DOE, and International Programs contributions.

\appendix
% Include all the relevant bib files.
% https://lsst-texmf.lsst.io/lsstdoc.html#bibliographies
\section{References} \label{sec:bib}
\renewcommand{\refname}{} % Suppress default Bibliography section
\bibliography{local,lsst,lsst-dm,refs_ads,refs,books}

% Make sure lsst-texmf/bin/generateAcronyms.py is in your path
%\section{Acronyms} \label{sec:acronyms}
%\input{acronyms.tex}
% If you want glossary uncomment below -- comment out the two lines above
%\printglossaries





\end{document}
